\section{Conclusion and Discussion}\label{sec:conclusion}

\ \quad 이 보고서에서는 simplex 데이터에 적용할 수 있는 nonparametric additive regression 모형을 이해하고, 다양한 미국의 지역별 데이터를 통해 다양한 변수에 따라 지역의 인종 비율이 어떻게 바뀌는 지를 분석하였다. 그리고 이 결과를 다양한 parametric 모형과 비교하여 성능을 확인하였다. 또한, 각 변수들의 변화에 따른 구체적인 인종 비율의 변화를 시각적으로 표현하였으며, 이는 Section \ref{sec:results}에서 보듯 해석 가능하고 흥미로운 분석 결과를 제공하였다.

한편 데이터의 처리 과정에서의 어려움이 다소 있었다. 우선 데이터를 얻고 취합하는 과정에서, 최대한 일치된 시기에 수집된 데이터를 사용하려고 하였으나 강수량 등의 데이터는 수집에 일정 기간이 필요하기 때문에 다소 한계가 있었다. 또한, 원래 데이터를 수집하는 지역 단위는 county로 하고자 하였으나 county별 인구수 등의 차이가 매우 심하고 일부 변수는 수집하는데 어려움이 있어, state 단위로 데이터를 수집하였고 이로 인해 데이터의 개수가 다소 적다는 문제가 발생하였다. 이에 더해 반응변수로 선정한 미국 인종 비율 데이터는 전체적으로 Caucasian(백인)의 비율이 높은 unbalanced 데이터이므로 적합에 어려움이 있었다.

Section \ref{sec:results}에서 알 수 있듯이 분석 대상인 미국 지역의 데이터에 대해 nonparametric additive 모형은 다른 parametric 모형과 validation error를 비교하였을 때 충분히 만족스럽게 적합되지는 못하였다. Section \ref{sec:methods}에서 언급했듯 nonparametric 모형의 경우 다른 모형에 비해 그 형태에 제약이 더 작은 대신 분석의 대상인 미국 지역 데이터에 대한 최적화가 충분하지 못했던 것이 하나의 이유라고 할 수 있다. 기존 분석에서는 Jeon and Park(2018)에서 제공한 소스코드를 사용하였는데, 이를 문제에 맞게 세부적으로 조절할 수 있다면 조금 더 개선할 수 있을 것으로 생각된다. 
