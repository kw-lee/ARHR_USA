\section{Introduction}
\label{sec:intro}

\ \quad 우리는 비모수함수추정론 강의를 통해 유클리드 공간 상의 자료(Euclidean data)에 smooth backfitting 알고리듬을 사용한 가법모형(additive regression model)의 적합, 그리고 이 알고리듬의 수렴성(convergence)\을 포함한 점근적 성질(asymptotic property)에 대해 학습하였다. 또한, \citet{jeon2018additive}을 통해 반응변수(response variable)를 힐베르트 공간 상의 자료(Hilbertian data)로 확장한 모형을 소개하고, 그 이론적 성질들을 다루었다. 
% 특히, \citet{jeon2018additive}의 논문을 통해 유클리드 공간 위의 자료에 대해 제안된 smoothing backfitting 알고리듬을 힐베르트 공간 위의 자료로 확장한 방법을 다루었다.

본 보고서에서는 수업시간에 다룬 모형을 실제 자료에 적용하고 그 결과를 살펴보고자 한다. 구체적으로는, \citet{jeon2018additive}에 소개된 여러 모형들을 통해 미국 지역별 인종 비율 자료를 분석한다. 미국은 다양한 인종이 살고 있는 나라이고, 각 주마다 기온, 강수량 등의 환경적인 요소들과 평균 소득 등의 생활 관련 요소들이 서로 다르다. 이러한 요소들이 인종 비율에 어떻게 연관되는지 파악하기 위해 가법모형을 적합하고, 각 설명변수의 변화에 따라 인종 비율이 어떻게 변하는 지를 시각적으로 나타내보고자 한다.

미국 지역별 인종 비율 자료는 $p$차원 simplex $S_1^p$ 위의 자료이다. $S_1^p$는 $p$ 차원 유클리드 공간의 부분공간 중 모든 원소의 합이 1이라는 조건을 만족하는 공간, 즉, 다음과 같은 공간을 의미한다
$$S_1^p = \Set{\bx = (x_1, \cdots, x_p) \in \mathbb{R}^p : x_i \geq 0,~i=1,2,\cdots, p,~\sum_{i=1}^p x_i =1}.$$
$S_1^p$에 적당한 연산을 부여하면 이 공간을 비유클리드(non-Euclidean) 힐베르트 공간으로 만들 수 있다. \citet{jeon2018additive} 또한, 이러한 공간 위의 자료로 한국의 선거 결과 자료를 분석한 결과를 본문에 소개하고 있다.

각 절은 다음과 같이 구성되어있다. Section \ref{sec:dat}에서는 분석에 사용할 자료에 대해 간략히 소개한다. Section \ref{sec:compositional}에서는 $p$차원 simplex $S_1^p$를 힐베르트 공간으로 나타내는 방법과 그러한 공간 위의 자료를 분석하기 위해 제안된 모형들을 소개한다. Section \ref{sec:results}에서는 소개한 모형들을 실제 자료에 적용한 결과를 다루며 Section \ref{sec:conclusion}에서 그 결과에 대한 해석과 제언을 다룬다.