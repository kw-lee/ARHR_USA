\section{Data}
\label{sec:dat}

% intro?
% 본 보고서에서는 미국 각 주별 인종 분포와 그 경향성을 설명할 수 있을 것이라 기대하는 변수들의 관계에 대해 알아보고자 한다. 구체적으로는, 각 주별 인종 비율과 각 주별 연령, 소득, 범죄율, 온도, 강수량의 관계에 대해 살펴보고자 한다. 

분석에 이용한 자료는 다양한 출처를 통해 수집하였다. 먼저, 각 주(State)별 인종 분포는 미국 인구조사국(Census Bureau)에서 제공하는 State Population by Characteristics: 2010-2018 자료\footnote{\url{https://www.census.gov/data/tables/time-series/demo/popest/2010s-state-detail.html}}를 바탕으로 계산하였다. 자료는 2010년부터 2018년까지 각 주별 특성별(성별, 인종, 나이 등) 인구수로 구성되어 있다. 자료에서는 인종을 (1) 백인과 그 혼혈, (2) 흑인과 그 혼혈, (3) 아시안과 그 혼혈, (4) 아메리카 원주민과 알래스카 원주민, 그리고 그들의 혼혈, (5) 하와이 원주민과 그 혼혈로 나누어 조사하였으며, 한 사람이 여러 인종의 혼혈인 경우, 그 사람을 해당되는 모든 항목에서 셈하고 있다. 여기서는 2017년 기준 자료를 사용하였으며, 분석과 시각화의 편의를 위하여 가장 많은 비율을 차지하는 백인과 그 혼혈($Y_1$), 흑인과 그 혼혈($Y_2$), 아시안과 그 혼혈($Y_3$)의 비율을 반응변수 $\bY = (Y_1, Y_2,Y_3) \in S_1^3$로 사용하였다.

각 주별 연령($X1$) 또한 State Population by Characteristics: 2010-2018 자료를 통해 계산한 2017년 기준 주별 연령 분포의 중앙값을 사용하였다.
% 나머지 얘들도 위와 같이 조금 더 구체적으로 해주셔야할 듯

% \ \quad 분석에 이용한 자료는 복수의 채널을 통하여 획득하였다. 먼저, 인종의 경우 2017년에 각 주 별 인종 분포를 (1) 백인과 그 혼혈, (2) 흑인과 그 혼혈, (3) 아시안과 그 혼혈, (4) 아메리카 원주민과 알래스카 원주민, 그리고 그들의 혼혈, (5) 하와이 원주민과 그 혼혈로 나누어 조사한 자료를 이용하였다. 이 자료에서는 한 사람이 여러 인종의 혼혈인 경우, 그 사람을 해당되는 모든 항목에서 셈하고 있다. 우리는 분석과 시각화의 편의를 위하여 가장 많은 비율을 차지하는 (1) 백인과 그 혼혈, (2) 흑인과 그 혼혈, (3) 아시안과 그 혼혈을 분석에 이용하였고, 세 인종의 합을 전체로 하여 비율을 다시 계산하고 사용하였다.

% 다음으로 소득의 경우 2017년 각 주 별 일년 간의 가계 소득 분포의 중앙값 자료를 사용하였고, 연령의 경우 2017년 각 주 별 연령 분포의 중앙값을 사용하였다. 또한, 범죄율의 경우 2017년 각 주 별 인구 10만명 당 강력 범죄의 빈도 자료를 사용하였다. 마지막으로, 기후의 경우, 다른 자료들과 달리 연간 편차가 심하다는 점을 고려하여, 1981년부터 2010년까지의 각 주 별 평균 기온 자료와 평균 강수량 자료를 사용하였다.

각 자료는 미국 각 주별 관측치와 워싱턴 D.C.의 관측치, 총 51개의 관측치로 이루어져 있다.