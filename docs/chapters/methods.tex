\section{Methods}\label{sec:methods}

\ \quad 반응변수 $Y = (Y_1, Y_2,Y_3)$는 유클리드 공간 상의 자료가 아닌 3차원 simplex $S_1^3$ 위의 자료이다. 이 절에서는 $p$차원 simplex $S_1^p$ 위의 자료(compositional data)를 분석하기 위해 제안된 모형들에 대해 간략히 소개한다. 

가장 간단한 모형으로, \textit{Multinomial logistic regression}
\begin{equation}\label{eqn:multilogis}
    \begin{gathered}
        \expt{\bY|\bX=\bx} = (\alpha_1(\bx), \alpha_2(\bx), \cdots, \alpha_p(\bx)), \\
        \alpha_1(\bx) = \frac{1}{1+\sum_{l=2}^p \exp(\beta_{l0}+\bbeta_l^T \bx)},~ \alpha_j(\bx) = \frac{\exp(\beta_{j0}+\bbeta_j^T)}{1+\sum_{l=2}^p \exp(\beta_{l0}+\bbeta_l^T \bx)},~ \text{for }j \geq 2
    \end{gathered}
\end{equation}
을 생각해볼 수 있다. 여기서 모수 $\beta_{j0}$, $\bbeta_j$의 추정으로 최소제곱 추정(least squares)이나 다음과 같이 정의되는 Kullback-Leibler divergence를 최소화하는 값을 사용할 수 있다.
$$\text{KL} = \sum_{i=1}^p \sum_{j=1}^p Y_{ij} \Log{Y_{ij}/\alpha_j(\bX_i)}.$$
각 모형은 R의 \texttt{Compositional} 패키지의 \texttt{ols.compreg}, \texttt{kl.compreg}를 이용해 적합할 수 있다. 해당 패키지에서는 Jensen-Shannon divergence, symmetric Kullback-Leibler divergence 등을 최소화하는 모형을 적합하는 함수들 또한 제공하고 있다.

$\bX$가 주어져 있을 때, $\bY$의 조건부 분포를 디리클레(Dirichlet) 분포로 가정한 \textit{Dirichlet regression model}
\begin{equation}\label{eqn:dirireg}
    \begin{gathered}
        \bY|\bX=\bx \sim \text{Dirichlet}(\phi \alpha_1(\bx), \phi \alpha_2(\bx),\cdots, \phi \alpha_p(\bx)),~\phi > 0
    \end{gathered}
\end{equation}
을 적합할 수도 있다. 이때 $\phi$, $\beta_{j0}$, $\bbeta_j$의 값의 추정은 최대가능도 추정(maximum likelihood estimation)을 이용한다. \texttt{Compositional} 패키지의 \texttt{diri.reg} 함수를 이용해 적합할 수 있다.

위의 모형들은 $Y_i=0$인 자료가 존재할 때, 모형이 제대로 추정되지 않는 문제가 발생한다. 이 경우 \citet{tsagris2015regression}가 제안한 \textit{Regression with compositional data using the $\alpha$-transformation} ($\alpha$-regression)\를 고려해볼 수 있다. $\alpha$-regression 모형은 \citet{tsagris2011data}가 제안한 $\alpha$-transform을 통해 자료를 변환하고, 변환된 자료에 적당한 모수적 모형을 적합한다. 이때, 조율 모수(tunning parameter) $|\alpha|<1$ 값은 cross validation을 통해 결정할 수 있다. $\alpha$-regression 모형은 \texttt{Compositional} 패키지의 \texttt{alfa.reg} 함수를 이용해 적합할 수 있다.

앞서 언급한 모형들은 모두 모수적 모형으로 일반적인 형태의 함수를 추정하기엔 제약이 있다. 이러한 문제를 해결할 수 있는 모형으로 \citet{jeon2018additive}이 제안한 \textit{Bochner smooth backfitting estimators} (B-SBF)을 고려해볼 수 있다. B-SBF 모형은 앞서 언급한 모형들에 비해 모형의 형태에 큰 제약을 두지 않는다. 이러한 특징으로 적합에는 앞서 언급한 모형들보다 더 많은 시간을 필요로 하지만 유연한 구조를 갖는 함수를 추정할 수 있는 강점을 갖는다. B-SBF 모형은 github repository \url{https://github.com/jeong-min-jeon/Add-Reg-Hilbert-Res}의 소스코드를 이용해 적합할 수 있다.