\section{Conclusion and Discussion}\label{sec:conclusion}

\ \quad 이 보고서에서는 simplex 위의 자료에 적용할 수 있는 비모수 가법모형을 이해하고, 다양한 미국의 지역별 자료를 통해 다양한 변수에 따라 지역의 인종 비율이 어떻게 바뀌는 지를 분석하였다. 그리고 이 결과를 다양한 모수적 모형과 비교하여 성능을 확인하였다. 또한, 각 변수들의 변화에 따른 구체적인 인종 비율의 변화를 시각적으로 표현하였으며, 이는 Section \ref{sec:visual}에서 보듯 해석 가능하고 흥미로운 분석 결과를 제공하였다.

Section \ref{sec:comparison}에서 알 수 있듯이 분석 대상인 미국 지역의 자료에 대해 비모수적 모형은 모수적 모형과 비교하였을 때 충분히 만족스럽게 적합되지 못하였다. Section \ref{sec:compositional}에서 언급했듯이, 비모수적 모형은 모수적 모형에 비해 그 형태에 제약이 더 작다는 강점을 가지나, 실제 모형(true model)이 모수적 모형일 때 과적합 등으로 인해 더 낮은 성능을 가질 수 있다.
%분석의 대상인 미국 지역 데이터에 대한 최적화가 충분하지 못했던 것이 하나의 이유라고 할 수 있다. 
본 보고서에서 사용한 자료의 크기가 충분히 크지 않았던 점을 고려해보면 이러한 이유로 인해 비모수적 모형이 적합 과정에서는 더 오랜 시간이 걸렸음에도 성능은 더 떨어지는 결과를 보였을 것이라 추측할 수 있다. 또한, 반응변수로 선정한 미국 인종 비율 자료는 전체적으로 백인의 비율이 높은 불균형한(unbalanced) 자료였는데, 그 결과 추정된 값이 simplex의 경계에 위치했을 때, ASPE 값이 극단적으로 높게 계산되는 문제가 발생했다. $\alpha$-regression은 적당한 변환을 통해 이러한 문제를 해결한 것으로 생각된다.

마지막으로 본 보고서의 발전 방향을 몇 가지 언급하고자 한다. 먼저, 더 나은 자료를 사용할 것을 제안한다. 기존 계획은 미국의 각 county 별 자료를 수집하는 것이었으나, 자료의 출처마다 관측치의 정확도의 차이가 심하고 분포가 매우 불균형하여 주 단위로 자료를 수집하였다. 향후의 분석에서는 더 많은 자료를 통해 계산한 안정적인 결과에 대한 비교를 제안한다. 다음으로, 자료에 맞는 모형의 최적화(fine-tunning)\를 제안한다. 기존 분석에서는 \citet{jeon2018additive}에서 제공하는 함수만을 사용하였는데, 이를 문제에 맞게 세부적으로 조절할 수 있다면 더 나은 결과를 얻을 수 있을 것으로 생각된다. 

% 우선 데이터를 얻고 취합하는 과정에서, 최대한 일치된 시기에 수집된 데이터를 사용하려고 하였으나 강수량 등의 데이터는 수집에 일정 기간이 필요하기 때문에 다소 한계가 있었다. 또한, 원래 데이터를 수집하는 지역 단위는 county로 하고자 하였으나 county별 인구수 등의 차이가 매우 심하고 일부 변수는 수집하는데 어려움이 있어, 주(state) 단위로 데이터를 수집하였고 이로 인해 데이터의 개수가 다소 적다는 문제가 발생하였다. 이에 더해 반응변수로 선정한 미국 인종 비율 데이터는 전체적으로 백인의 비율이 높은 unbalanced 데이터이므로 적합에 어려움이 있었다. 
% 기존 분석에서는 \citet{jeon2018additive}에서 제공한 소스코드를 사용하였는데, 이를 문제에 맞게 세부적으로 조절할 수 있다면 조금 더 개선할 수 있을 것으로 생각된다. 